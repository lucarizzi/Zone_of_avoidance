%%
%% Beginning of file 'sample.tex'
%%
%% Modified 2005 December 5
%%
%% This is a sample manuscript marked up using the
%% AASTeX v5.x LaTeX 2e macros.

%% The first piece of markup in an AASTeX v5.x document
%% is the \documentclass command. LaTeX will ignore
%% any data that comes before this command.

%% The command below calls the preprint style
%% which will produce a one-column, single-spaced document.
%% Examples of commands for other substyles follow. Use
%% whichever is most appropriate for your purposes.
%%
%%\documentclass[12pt,preprint]{aastex}

%% manuscript produces a one-column, double-spaced document:

\documentclass[12,manuscript,usenatbib]{aastex}
\usepackage{amsmath}

%% preprint2 produces a double-column, single-spaced document:

%% \documentclass[preprint2]{aastex}

%% Sometimes a paper's abstract is too long to fit on the
%% title page in preprint2 mode. When that is the case,
%% use the longabstract style option.

%% \documentclass[preprint2,longabstract]{aastex}

%% If you want to create your own macros, you can do so
%% using \newcommand. Your macros should appear before
%% the \begin{document} command.
%%
%% If you are submitting to a journal that translates manuscripts
%% into SGML, you need to follow certain guidelines when preparing
%% your macros. See the AASTeX v5.x Author Guide
%% for information.

\newcommand{\vdag}{(v)^\dagger}
\newcommand{\myemail}{lrizzi@keck.hawaii.edu}

%% You can insert a short comment on the title page using the command below.

\slugcomment{Not to appear in Nonlearned J., 45.}

%% If you wish, you may supply running head information, although
%% this information may be modified by the editorial offices.
%% The left head contains a list of authors,
%% usually a maximum of three (otherwise use et al.).  The right
%% head is a modified title of up to roughly 44 characters.
%% Running heads will not print in the manuscript style.

\shorttitle{Distance to ALFAZOA}
\shortauthors{Rizzi et al.}

%% This is the end of the preamble.  Indicate the beginning of the
%% paper itself with \begin{document}.

\begin{document}

%% LaTeX will automatically break titles if they run longer than
%% one line. However, you may use \\ to force a line break if
%% you desire.

\title{The Distance to the ALFAZOAJ1952+1428 and KK246 galaxies.}

%% Use \author, \affil, and the \and command to format
%% author and affiliation information.
%% Note that \email has replaced the old \authoremail command
%% from AASTeX v4.0. You can use \email to mark an email address
%% anywhere in the paper, not just in the front matter.
%% As in the title, use \\ to force line breaks.

\author{L. Rizzi}
\affil{W. M. Keck Observatory, Kamuela, HI 96743}
\email{lrizzi@keck.hawaii.edu}

\and

\author{R. B. Tully}
\affil{Institute for Astronomy, University of Hawaii, Honolulu, HI 96822}

\and


\author{E. J. Shaya}
\affil{University of Maryland, Astronomy Department, College Park, MD 20743}


%% Mark off your abstract in the ``abstract'' environment. In the manuscript
%% style, abstract will output a Received/Accepted line after the
%% title and affiliation information. No date will appear since the author
%% does not have this information. The dates will be filled in by the
%% editorial office after submission.

\begin{abstract}
TBD
\end{abstract}

%% Keywords should appear after the \end{abstract} command. The uncommented
%% example has been keyed in ApJ style. See the instructions to authors
%% for the journal to which you are submitting your paper to determine
%% what keyword punctuation is appropriate.

\keywords{TBD}

%% From the front matter, we move on to the body of the paper.
%% In the first two sections, notice the use of the natbib \citep
%% and \citet commands to identify citations.  The citations are
%% tied to the reference list via symbolic KEYs. The KEY corresponds
%% to the KEY in the \bibitem in the reference list below. We have
%% chosen the first three characters of the first author's name plus
%% the last two numeral of the year of publication as our KEY for
%% each reference.


%% Authors who wish to have the most important objects in their paper
%% linked in the electronic edition to a data center may do so by tagging
%% their objects with \objectname{} or \object{}.  Each macro takes the
%% object name as its required argument. The optional, square-bracket 
%% argument should be used in cases where the data center identification
%% differs from what is to be printed in the paper.  The text appearing 
%% in curly braces is what will appear in print in the published paper. 
%% If the object name is recognized by the data centers, it will be linked
%% in the electronic edition to the object data available at the data centers  
%%
%% Note that for sources with brackets in their names, e.g. [WEG2004] 14h-090,
%% the brackets must be escaped with backslashes when used in the first
%% square-bracket argument, for instance, \object[\[WEG2004\] 14h-090]{90}).
%%  Otherwise, LaTeX will issue an error. 

\section{Introduction}

TBD

\section{Observations}

%% In a manner similar to \objectname authors can provide links to dataset
%% hosted at participating data centers via the \dataset{} command.  The
%% second curly bracket argument is printed in the text while the first
%% parentheses argument serves as the valid data set identifier.  Large
%% lists of data set are best provided in a table (see Table 3 for an example).
%% Valid data set identifiers should be obtained from the data center that
%% is currently hosting the data.
%%
%% Note that AASTeX interprets everything between the curly braces in the 
%% macro as regular text, so any special characters, e.g. "#" or "_," must be 
%% preceded by a backslash. Otherwise, you will get a LaTeX error when you 
%% compile your manuscript.  Special characters do not 
%% need to be escaped in the optional, square-bracket argument.

TBD
%% In this section, we use  the \subsection command to set off
%% a subsection.  \footnote is used to insert a footnote to the text.

%% Observe the use of the LaTeX \label
%% command after the \subsection to give a symbolic KEY to the
%% subsection for cross-referencing in a \ref command.
%% You can use LaTeX's \ref and \label commands to keep track of
%% cross-references to sections, equations, tables, and figures.
%% That way, if you change the order of any elements, LaTeX will
%% automatically renumber them.

%% This section also includes several of the displayed math environments
%% mentioned in the Author Guide.

\section{Tip of the Red Giant Branch}

We derive the magnitude of the TRGB following the method described in \citet{2006AJ....132.2729M}. A parametric luminosity function is defined as:

\begin{equation}
 \psi = \left\{ \begin{array}{lr}
                 10^{a(m-m_{TRGB})+b}, & m - m_{TRGB} \ge 0, \\
                 10^{c(m-m_{TRGB})},   & m - m_{TRGB} < 0.
                \end{array} \right.
\end{equation}

This  function is then convolved with the completeness, uncertainty and photometric bias observational effects:

\begin{equation}
 \varphi(m) = \int \psi(m') \rho(m') e(m|m')dm',
\end{equation}

where $\rho(m)$ is the completeness as a function of magnitude and $e(m|m')$ is the error distribution function, both derived from artificial star tests. The error distribution function contains photometric uncertainty and photometric bias:

\begin{equation}
e(m|m') = \frac{1}{\sqrt{2\pi}\sigma(m')} \exp \left\{ - \frac{[m - \bar{m}(m')]^2}{2\sigma^2(m')} \right\},
\end{equation}

where $\sigma(m)$ is the uncertainty and $(\bar m)$ is the bias. 

To avoid uncertainties related to the choice of magnitude bins in determining the observed luminosity function, we construct a smoothed luminosity function following the prescription of \citet{1996ApJ...461..713S}. In summary, the discretely distributed stellar magnitudes are replaced by their corresponding Gaussian, following the expression:
\begin{equation}
\phi(m) = \sum_{i=1}^{N} \frac{1}{\sqrt{2\pi}\sigma_{i}} \exp \left\{ - \frac{(m_i-m)^2}{2\sigma_i^2}\right\},
\end{equation}

The first guess for the position of the TRGB is obtained by applying a Sobel edge-detection filter to the smoothed luminosity function.


\section{Equations used}
\subsection{Reddening}
Equations are from \citet{2011ApJ...737..103S}.

\begin{equation}
A_{F814W} = 1.526 E(B-V)
\end{equation}
\begin{equation}
A_{F606W} = 2.488 E(B-V)
\end{equation}
\begin{equation}
A_{F110W} = 0.881E(B-V)
\end{equation}
\begin{equation}
A_{F160W} = 0.512 E(B-V)
\end{equation}

Derived from previous equations:

\begin{equation}
E(F110W-F160W) = 0.369 E(B-V)
\end{equation}
\begin{equation}
E(F606W-F814W) = 0.962 E(B-V)
\end{equation}

\subsection{Absolute magnitudes of TRGB}
Equations derived from \cite{2007ApJ...661..815R}.

\begin{equation}
M_{F814W} = -4.06 + 0.20 [(F606W-F814W) -1.23]
\end{equation}

Equations derived from \citet{2014AJ....148....7W}.
\begin{equation}
M_{F110W}=\left\{\begin{matrix}
-5.02 -1.41 \times [(F110W-F160W)-0.95], F110W-F160W \leqslant 0.95,
\\ 
-5.02 -2.81 \times[(F110W-F160W)-0.95], F110W-F160W > 0.95
\end{matrix}\right.
\end{equation}

\begin{equation}
M_{F160W}=\left\{\begin{matrix}
-5.97 -2.41 \times [(F110W-F160W)-0.95], F110W-F160W \leqslant 0.95,
\\ 
-5.97 -3.81 \times[(F110W-F160W)-0.95], F110W-F160W > 0.95
\end{matrix}\right.
\end{equation}




\section{Results for ALFAZOAJ1952+1428}

\begin{equation}
F814W_{TRGB}=25.95 \pm 0.09 \\
\end{equation}
\begin{equation}
(F814W-F606W)_{TRGB}=1.32 \pm 0.08
\end{equation}
\begin{equation}
E(B-V): 0.28
\end{equation}
\begin{equation}
F814W_{TRGB,0} = 25.53
\end{equation}
\begin{equation}
F814W_{Absolute} = -4.09
\end{equation}
\begin{equation}
(m-M)_0=29.62
\end{equation}

\begin{figure}
\epsscale{.80}
\plotone{cmd_ALPHAZOA.eps}
\caption{Color magnitude diagram.\label{cmd1}}
\end{figure}

\begin{figure}
\epsscale{.80}
\plotone{lf_ALPHAZOA.eps}
\caption{Measurement of the tip.\label{lf1}}
\end{figure}

\begin{figure}
\epsscale{.80}
\plotone{reddening_ALPHAZOA.eps}
\caption{Reddening measurements.\label{reddening1}}
\end{figure}

\subsection{Results for KK246}

\begin{equation}
F110W_{TRGB}=24.82 \pm 0.06
\end{equation}
\begin{equation}
F160W_{TRGB}=23.86 \pm 0.04
\end{equation}
\begin{equation}
(F110W-F160W)_{TRGB}=0.86 \pm 0.09
\end{equation}
\begin{equation}
E(B-V): 0.67
\end{equation}
\begin{equation}
F110W_{TRGB,0} = 24.23
\end{equation}
\begin{equation}
F160W_{TRGB,0} = 23.52
\end{equation}
\begin{equation}
F110W_{Absolute} = -4.50
\end{equation}
\begin{equation}
F160W_{Absolute} = -5.28
\end{equation}
\begin{equation}
(m-M)_{F110W,0}=28.73
\end{equation}
\begin{equation}
(m-M)_{F160W,0}=28.79
\end{equation}

\begin{figure}
\epsscale{.80}
\plottwo{cmd_KK246_J.eps} {cmd_KK246_H.eps}
\caption{Color magnitude diagram.\label{cmd2}}
\end{figure}

\begin{figure}
\epsscale{.80}
\plotone{lf_KK246_J.eps}
\caption{Measurement of the tip in F110W.\label{lf2_1}}
\end{figure}

\begin{figure}
\epsscale{.80}
\plotone{lf_KK246_H.eps}
\caption{Measurement of the tip in F160W.\label{lf2_2}}
\end{figure}


\begin{figure}
\epsscale{.80}
\plotone{reddening_KK246.eps}
\caption{Reddening measurements.\label{reddening2}}
\end{figure}



\bibliography{alfazoa}
\bibliographystyle{apj}


\end{document}

%%
%% End of file `sample.tex'.
